\documentclass[12pt]{article}
\usepackage{lingmacros}
\usepackage{tree-dvips}
\title{Tomasulo-Softcore Library}
\author{Yihao Liu \& Stephen Longofono}
\begin{document}
\maketitle
\section*{Introduction}
This document outlines the Tomasulo-Softcore library, a Python library written to model a single-core software CPU based on Tomasulo's Algorithm.  The source code is presented as an open-source library, and is available on Github.com under the MIT license.  This library was developed as an academic exercise for the graduate-level computer architecture course at the University of Pittsburgh in Pennsylvania, U.S.A.

The library demonstrates all of the major facets of the algorithm, including register renaming, out-of-order execution, memory disambiguation, speculative execution.

Please reference the Github repository for the most up-to-date code, instructions, and test cases.  The repository is hosted at: \\
\textbf{https://www.github.com/SLongofono/Tomasulo-Softcore.git/}

\section{Building \& Execution}
\subsection{System Requirements}
The library was developed for Python 3.6+.  No external libraries were used to develop the core library.  The project was developed and tested in a Linux environment, in a directory with write and execute privileges.  Note that the output files will be placed in the same directory as the input files passed in; in order to correctly generate an output file, the user must have write privileges to the directory with the input files.

Included in the repository is a shell script used to generate UML diagrams from the class hierarchy.  If you choose to modify or extend the library, the script can be run again from the root directory of the repository to generate fresh images of the UML class diagrams.  Running this script requires Pylint 2.2.1+, but is not otherwise necessary for the use of the core library.

\subsection{Build Instructions}
There are no build instructions for the library, as it is a standalone Python module.  The entry point to execution is Tomasulo.py.  To start the library and print usage instructions, simply invoke your Python interpreter with Tomasulo.py as an argument from the command line.

\subsection{Running Test Inputs}
All test inputs are passed as command line arguments.  The test inputs all follow a rigid and ordered format, as demonstrated by the file "example\_input.txt" in the "testCases/" directory of the repository root.  For convenience, all of our test cases are located in this "textCases/" directory.  To execute the library on a test input file, type:
\begin{verbatim}
    <Python3.6> Tomasulo.py <filepath>
\end{verbatim}
where Python3.6 is the CLI invoker for your Python 3 interpreter, and filepath is the absolute filepath of the desired test input.

\subsection{Defining System Inputs}
The system inputs follow a rigid and specifically ordered format.  All inputs should be UTF-8 compliant, printable text format files.  The specifications are presented in order below:
\begin{itemize}
    \item The first line of the file is ignored.  In our sample inputs, the first line is used to label the table columns that the following 4 lines compose.
    \item The next four lines begin with a label, followed by a space, and then 4 entries of either a natural number or a space (only permissible in the third position).  The labels are "Integer adder", "FP adder", "FP multiplier", and Load/store unit".
    \item The next line begins with the label "ROB entries = ", followed by a natural number.
    \item The next line begins with "CDB buffer entries = ", followed by a natural number.
    \item The next line is optional.  It consists of comma-delineated patterns of the form A=B.  The pattern A is the letter "R" or the letter "F" concatenated with exactly one of the numbers 0-31.  The pattern B is an integer value which can be represented in 32 bits.
    \item The next line is optional.  It consists of comma-delineated patterns of the form Meme[A]=B.  The pattern A is a single natural number between 0 and 255 inclusive.  The pattern B is an integer or floating point value which can be represented in 64 bits.
    \item The next line is blank.  This line is not optional.
    \item The remainder of the lines in the file are instructions.  All instructions are represented as shown in the rubric.  Replace immediate values and offsets by their integer equivalents.  In the case of a branch instruction, express the offset in terms of instructions to move relative to PC-next.
    \item The last line of the file should be the last instruction.
\end{itemize}

\subsection{System Outputs}
The library will output execution information in two ways; during execution, progress and the contents of various system components are displayed to \textbf{stdout}.  Upon completion, an output file will be written to the directory containing the input file.  The output file will be the same name as the input file, with "\_output" appended just before the file extension.

\section{Verification \& Testing}
In order to verify a proper and complete implementation of Tomasulo's algorithm, we developed a series of progressively more complex test inputs.  All of the test cases used are provided in the repository under the "testCases/" directory.  For convenience, they are repeated here along with a short description of which feature they verify.

\subsection{simple1.txt}
This test was designed to test a pair of independent integer ALU instructions, as a verification that our system could correctly process both types of Integer ALU instructions in isolation.  It also serves as a verification of the general flow of the program, excluding the memory stage.

\subsubsection*{Input}
\begin{verbatim}
  # of rs Cycles in EX  Cycles in Mem # of FUs
Integer adder 2 1   1
FP adder  3 3   1
FP multiplier 2 20    1
Load/store unit 3 1 4 1
ROB entries = 128
CDB buffer entries = 1
R1=10,R2=20,R3=30,F2=30.1
Mem[4]=1,Mem[8]=2,Mem[12]=3.4

ADD R1,R2,R3
SUB R4,R2,R3
\end{verbatim}

\subsubsection*{Output}
\begin{verbatim}
================================Instruction Completion Table====================
ID	| IS		 EX		 MEM		 WB		 COM
0	| 0		 1		 None		 2		 3
1	| 1		 2		 None		 3		 4

================================Integer ARF=====================================
R0 : 0              R1 : 50             R2 : 20             R3 : 30             
R4 : -10            R5 : 0              R6 : 0              R7 : 0              
R8 : 0              R9 : 0              R10: 0              R11: 0              
R12: 0              R13: 0              R14: 0              R15: 0              
R16: 0              R17: 0              R18: 0              R19: 0              
R20: 0              R21: 0              R22: 0              R23: 0              
R24: 0              R25: 0              R26: 0              R27: 0              
R28: 0              R29: 0              R30: 0              R31: 0              

================================Floating Point ARF==============================
F0 : 0.000000                           F1 : 0.000000                           
F2 : 30.100000                          F3 : 0.000000                           
F4 : 0.000000                           F5 : 0.000000                           
F6 : 0.000000                           F7 : 0.000000                           
F8 : 0.000000                           F9 : 0.000000                           
F10: 0.000000                           F11: 0.000000                           
F12: 0.000000                           F13: 0.000000                           
F14: 0.000000                           F15: 0.000000                           
F16: 0.000000                           F17: 0.000000                           
F18: 0.000000                           F19: 0.000000                           
F20: 0.000000                           F21: 0.000000                           
F22: 0.000000                           F23: 0.000000                           
F24: 0.000000                           F25: 0.000000                           
F26: 0.000000                           F27: 0.000000                           
F28: 0.000000                           F29: 0.000000                           
F30: 0.000000                           F31: 0.000000                           


================================Memory Unit=====================================
Word 04: 1.000000                       Word 08: 2.000000                       
Word 12: 3.400000
\end{verbatim}

\subsection{simple2.txt}
This test was designed to test a pair of independent floating point ALU instructions, as a verification that our system could correctly process both types of floating point ALU instructions in isolation.  It also serves as a verification of the general flow of the program, excluding the memory stage.

\subsubsection*{Input}
\begin{verbatim}
    
\end{verbatim}

\subsubsection*{Output}
\begin{verbatim}
    
\end{verbatim}


\subsection{simple3.txt}
This test was designed to test a pair of independent floating point multiplier instructions, as a verification that our system could correctly process floating point multiplier instructions in isolation.

\subsubsection*{Input}
\begin{verbatim}
    
\end{verbatim}

\subsubsection*{Output}
\begin{verbatim}
    
\end{verbatim}

\subsection{simple4.txt}
This test case was designed to test the load and store instructions in isolation.  It also serves as a verification of the general flow of the program, excluding the branch unit.

\subsubsection*{Input}
\begin{verbatim}
    
\end{verbatim}

\subsubsection*{Output}
\begin{verbatim}
    
\end{verbatim}

\subsection{simple5.txt}
This test case was designed to test the branch unit with a single branch instruction.  It serves to test the simplest functions of the branch unit, including storing copies of the RAT and correct branch prediction.

\subsubsection*{Input}
\begin{verbatim}
    
\end{verbatim}

\subsubsection*{Output}
\begin{verbatim}
    
\end{verbatim}

\subsection{simple6.txt}
This test case was designed to test corner cases of the system configuration.  This test will issue instructions to saturate the capacity of all functional units by filling the output buffers and the reservation stations.  The goal is to prove that the library can make forward progress in the face of a demanding load.

\subsubsection*{Input}
\begin{verbatim}
    
\end{verbatim}

\subsubsection*{Output}
\begin{verbatim}
    
\end{verbatim}

\subsection{simple7.txt}
This test case was designed to verify the absence of WAW hazards, a behavior that the algorithm was designed to remove.

\subsubsection*{Input}
\begin{verbatim}
    
\end{verbatim}

\subsubsection*{Output}
\begin{verbatim}
    
\end{verbatim}

\subsection{simple8.txt}
This test case was designed to verify the absence of WAR hazards, a behavior that the algorithm was designed to remove.

\subsubsection*{Input}
\begin{verbatim}
    
\end{verbatim}

\subsubsection*{Output}
\begin{verbatim}
    
\end{verbatim}

\subsection{simple9.txt}
This test case was designed to verify that the system is capable of handling RAW dependencies.  Several truly dependent instructions are issued for each of the types of functional units, to test the ability to complete execution as prescribed by Tomasulo's algorithm.

\subsubsection*{Input}
\begin{verbatim}
    
\end{verbatim}

\subsubsection*{Output}
\begin{verbatim}
    
\end{verbatim}

\subsection{simple10.txt}
This test case was designed to verify correct operation of the memory unit.  Specifically, it tests that the memory unit is able to disambiguate dependencies among loads and stores, and ensure both correctness and precision of execution in the memory unit.

\subsubsection*{Input}
\begin{verbatim}
    
\end{verbatim}

\subsubsection*{Output}
\begin{verbatim}
    
\end{verbatim}

\subsection{simple11.txt}
This test case was designed to induce failure (exceptions) in the case of illegal instructions.  This includes writing to read-only registers, malformed instruction names, malformed register names, and improper types in the instruction input.

\subsubsection*{Input}
\begin{verbatim}
    
\end{verbatim}

\subsubsection*{Output}
\begin{verbatim}
    
\end{verbatim}

\subsection{complex1.txt}
This test case was designed to test a more true-to-life program, involving all the instructions in the ISA.  Dependencies exist among the instructions, and there are a sufficient number of instructions to induce a few stalls due to architecture.

\subsubsection*{Input}
\begin{verbatim}
    
\end{verbatim}

\subsubsection*{Output}
\begin{verbatim}
    
\end{verbatim}

\subsection{complex2.txt}
This test case was designed to test the ability of the branch unit to handle a failed speculation in isolation.  The predicted branch target is a group of long-latency instructions, ensuring that the branch unit is able to properly recover and entirely remove the speculative results from the data path.

\subsubsection*{Input}
\begin{verbatim}
    
\end{verbatim}

\subsubsection*{Output}
\begin{verbatim}
    
\end{verbatim}

\subsection{complex3.txt}
This test case was designed to further test the ability of the branch unit to handle failed speculation.  Two mis-predictions in a row are induced, in order to prove that the branch unit can properly back out speculations from two execution paths.

\subsubsection*{Input}
\begin{verbatim}
    
\end{verbatim}

\subsubsection*{Output}
\begin{verbatim}
    
\end{verbatim}

\subsection{complex4.txt}
This test case was designed to further test the ability of the branch unit to handle failed speculation.  Three branches are used in succession, with the first and third mis-predicted and the middle branch correctly predicted.  We chose this test because we are sadomasochists.

\subsubsection*{Input}
\begin{verbatim}
    
\end{verbatim}

\subsubsection*{Output}
\begin{verbatim}
    
\end{verbatim}

\subsection{complex5.txt}
This test case was designed to test the limits of the behavior of both the branch unit and the system.  A dependency is induced in the branch condition, followed by many instructions at the predicted target.  The branch condition is mis-predicted, and the result is a more onerous recovery than in previous tests.  

\subsubsection*{Input}
\begin{verbatim}
    
\end{verbatim}

\subsubsection*{Output}
\begin{verbatim}
    
\end{verbatim}

\subsection{complex6.txt}
This test case was designed to test our model of system memory (RAM).  Here, we write a series of patterns to memory to verify that loads and stores are executing in the correct order, and that data is written to the correct location.

\subsubsection*{Input}
\begin{verbatim}
    
\end{verbatim}

\subsubsection*{Output}
\begin{verbatim}
    
\end{verbatim}

\subsection{complex7.txt}
This final test case was designed to show that our system can do real work.  We execute a recursive version of a Fibonacci numbers procedure to induce a deep chain of RAW dependencies.

\subsubsection*{Input}
\begin{verbatim}
    
\end{verbatim}

\subsubsection*{Output}
\begin{verbatim}
    
\end{verbatim}

\section{Division of Labor}
\subsection{Responsibilities}
We developed our code and test cases together.  Stephen was more experienced with Python, so for productivity's sake he wrote much of the code.  Yihao helped by developing the general flow of the program, writing effective test cases, identifying areas for improvement in the design, and general debugging.

\subsection{Code Contribution}
Yihao developed the floating-point functional units and the memory unit.  Stephen developed the other main components of the library.  Stephen wrote the first pass of the five execution stages, and then worked with Yihao to debug them as each test case was visited.  More specific details on the day-to-day code contributions are available in the commit history of the Github repository.

\subsection{Test Contribution}
Stephen wrote the simpler test cases as a litmus test for the first pass of the data path.  Stephen helped come up with the ideas for the more exhaustive tests.  Yihao wrote the bulk of the test input files.  Both team members completed all tests by hand to verify that our expected results were consistent.

\subsection{Meetings \& Participation}
Since there were only two members on our team, it was easy to work together and hold productive meetings.  We spent several weekends coding side-by-side, and maintained effective communication throughout the project.

\end{document}
